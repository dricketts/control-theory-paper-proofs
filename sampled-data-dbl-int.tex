\documentclass[12pt]{article}

\usepackage{amsmath}
\usepackage{amsthm}

\newtheorem{theorem}{Theorem}

\title{State constraints for sampled-data systems}
\author{Daniel Ricketts}

\newcommand{\vecbold}[1]{\boldsymbol{#1}}
\newcommand{\umin}{u_{min}}
\newcommand{\p}{\gamma}
\newcommand{\q}{\alpha}
\newcommand{\state}{\mathcal{R}^n}

\begin{document}
\maketitle

\section{Exponential barrer functions for sampled-data systems}
Consider a sampled-data system, modeled by:
\[\dot{\vecbold{x}}(t) = f(\vecbold{x}(t),\vecbold{x}(t_k))\qquad\forall t \in [t_k, t_{k+1})\]
where $\vecbold{x}(t) \in \state$ and $0 = t_0 < t_1 < \ldots < t_k < \ldots$ are the sampling instants.

We would like to prove state invariants for this system. The idea is to
decompose this task into two pieces:
\begin{itemize}
\item First, we prove that intersample states are related in some way to
  the last sampled state. For example, in the double integrator, if we know
  that $|u| \leq u_{max}$ and we know that samples are at most $T$ apart,
  then the velocity of an intersample state differs by at most $u_{max}\cdot T$
  from the last sampled velocity.
\item Second, we prove, as before, that the time derivative of $B$ is at
  most proportional to $B$. However, the time derivative of $B$ is a
  function of both the current state and the last sampled state. Since
  we've proven that these two states are related in a certain way, we can
  assume that relation when proving the property about the time
  derivative. For the double integrator, we can assume that the intersample
  velocity differs by at most $u_{max}\cdot T$ when reasoning about the
  time derivative of $B$.
\end{itemize}

This intuition is formalized in Theorem~\ref{thm:exp-barrier-sampled}. In
this theorem, we use the fact that $\dot{B}(\vecbold{x}) = \nabla B \bullet
f(\vecbold{x},\vecbold{x_k})$. In the non-sampled data version of the
theorem, it was not necessary to expand this definition. However, it is
necessary here because we need to make explicit that the current state and
sampled state are related.

\begin{theorem}
If there exists a function $B \in \state \rightarrow \mathcal{R}$, some constant $\lambda \in \mathcal{R}$, and a state relation $S \subseteq \state \times \state$ such that
\begin{align}
\label{thm:exp-barrier-sampled:rel}
\forall k \in \mathcal{N}, t \in [t_k, t_{k+1}) : (\vecbold{x}(t),\vecbold{x}(t_k)) \in S \\
\label{thm:exp-barrier-sampled:dt}
\forall (\vecbold{x}, \vecbold{x_k}) \in S : \nabla B \bullet f(\vecbold{x},\vecbold{x_k}) \leq \lambda B(\vecbold{x})
\end{align}
then the set $\{\vecbold{x} \in \mathcal{R}^n : B(\vecbold{x}) \leq 0\}$ is forward invariant.
\label{thm:exp-barrier-sampled}
\end{theorem}

\section{Example: Single integrator}
Consider the sampled data version of the single integrator:
\[\dot{x}(t) = u(x(t_k)) \qquad \forall t \in [t_k, t_{k+1})\]
where $x(t) \in \mathcal{R}$, $0 = t_0 < t_1 < \ldots < t_k < \ldots$ are
the sampling instants, and $\forall k, t_{k+1} - t_k \leq T$ for some
constant $T > 0$.

We would like to prove that if $|u(x)| \leq u_{max}$ for some constant
$u_{max} > 0$ and $u(x) \leq -\lambda(x + u_{max}\cdot T)$, then $x \leq 0$
is forward invariant.

\paragraph{Intersample behavior}
First we find a state relation $S$ to characterize the intersample
behavior. Using the bound $|u(x)| \leq u_{max}$ and the intersample time
bound $T$, we can prove that the following relation bounds the difference
between a state and the most recent sample:
\[S = \{(x,x_k) : |x - x_k| \leq u_{max}\cdot T\}\]
This satisfies condition~\ref{thm:exp-barrier-sampled:rel} of
Theorem~\ref{thm:exp-barrier-sampled}.

\paragraph{Time derivative of barrier function}
Second, we need to prove condition~\ref{thm:exp-barrier-sampled:dt} of
Theorem~\ref{thm:exp-barrier-sampled} for the barrier function $B(x) = x$:
\[\nabla B \bullet f(\vecbold{x},\vecbold{x_k}) = u(x_k) \leq -\lambda x\]
This follows from the fact that $u(x_k) \leq -\lambda(x_k + u_{max}\cdot
T)$ and $(x,x_k) \in S$, i.e. $|x - x_k| \leq u_{max}\cdot T$.

Note that this approach leads to conservative inputs for certain cases
because the time derivative bound needs to hold over the entire sampling
interval. For the single integrator, this is unnecessary - it would be
sufficient for the input to violate this constraint at the end of the
interval as long as it satisfies the constraint with some buffer at the
beginning. However, if the dynamics of the system are slow relative to the
sampling time, then this conservativeness should be acceptable.

%% \section{Robust exponential barrier functions}
%% This section will present a version of the exponential barrier function
%% theorem (Theorem~\ref{thm:exp-barrier-robust}) for systems under
%% disturbances. The proof of that theorem will use the original barrier
%% function theorem (Theorem~\ref{thm:exp-barrier}), so I present it here for
%% reference:

%% \begin{theorem}
%% For a system of the form $\dot{\vecbold{x}} = f(\vecbold{x},t)$, if there exists a function $B \in \mathcal{R}^n \rightarrow \mathcal{R}$ and some constant $\lambda \in \mathcal{R}$ such that
%% \[\dot{B}(\vecbold{x}) \leq \lambda B(\vecbold{x})\]
%% then the set $\{\vecbold{x} \in \mathcal{R}^n : B(\vecbold{x}) \leq 0\}$ is forward invariant.
%% \label{thm:exp-barrier}
%% \end{theorem}

%% Now consider a system of the form:
%% \begin{align}
%% \dot{\vecbold{x}} = f(\vecbold{x} + d(t))
%% \label{eqn:sys}
%% \end{align}
%% where $x\in\mathcal{R}^n$ and $d \in \mathcal{R} \rightarrow \mathcal{R}^n$.

%% \begin{theorem}
%% If there exists a function $B \in \mathcal{R}^n \rightarrow \mathcal{R}$, a class-$\mathcal{K}$ function $\alpha$, and constants $\lambda, D \geq 0$ such that
%% \begin{align}
%% \forall \vecbold{x} \in \mathcal{R}^n : \dot{B}(\vecbold{x}) \leq -\lambda B(\vecbold{x}) \\
%% \forall \vecbold{v}, \vecbold{w} \in \mathcal{R}^n : B(\vecbold{v} + \vecbold{w}) \geq B(\vecbold{v}) - \alpha(\|\vecbold{w}\|) \\
%% \forall t \in R : \|d(t)\| \leq D
%% \end{align}
%% then the set $\{\vecbold{x} \in \mathcal{R}^n : B(\vecbold{x}) \leq \alpha(D)\}$ is forward invariant.
%% \label{thm:exp-barrier-robust}
%% \end{theorem}

%% \begin{proof}
%% \begin{align*}
%% \frac{d[B(\vecbold{x}) - \alpha(D)]}{dt} &= \frac{dB(\vecbold{x})}{d\vecbold{x}}f(\vecbold{x} + d(t)) \\
%%                                          &\leq -\lambda B(\vecbold{x} + d(t)) \\
%%                                          &\leq -\lambda \left(B(\vecbold{x}) - \alpha(\|d(t)\|)\right) \\
%%                                          &\leq -\lambda \left(B(\vecbold{x}) - \alpha(D)\right)
%% \end{align*}
%% By Theorem~\ref{thm:exp-barrier} with barrier function $B(\vecbold{x}) -
%% \alpha(D)$, the above inequality implies that the set $\{\vecbold{x} \in
%% \mathcal{R}^n : B(\vecbold{x}) - \alpha(D) \leq 0\}$ is forward invariant.
%% \end{proof}

\end{document}
