\documentclass[12pt]{article}

\usepackage{amsmath}
\usepackage{mathtools}

\newtheorem{theorem}{Theorem}

\title{Double integrator with state and input constraints}
\author{Daniel Ricketts}

\newcommand{\vecbold}[1]{\boldsymbol{#1}}
\newcommand{\umin}{u_{min}}
\newcommand{\p}{\gamma}
\newcommand{\q}{\alpha}

\begin{document}
\maketitle

\section{Problem}
Consider the following double integrator:
\begin{align}
\begin{split}
\dot{x} &= v \\
\dot{v} &= u
\end{split}
\label{sys1}
\end{align}

where
\[
u \leq
\begin{cases}
-\q(v - \sqrt{2\umin(-x + \frac{\umin}{2\p^2})}) - \frac{\umin v}{\sqrt{2\umin(-x + \frac{\umin}{2\p^2})}} & \text{if } x < \frac{\umin}{\p^2}\\
-(\q + \p) v -\q\p x & \text{otherwise}
\end{cases}
\]

We want to prove that this system satisfies the state constraint $x \leq 0$
while obeying the input constraint $u \geq \umin$ where $\umin < 0$, under
the assumption that $x \leq 0$ and $B(x,v) \leq 0$ hold on the initial
state, where

\[B(x,v) =
\begin{cases}
v - \sqrt{2\umin(-x + \frac{\umin}{2\p^2})} & \text{if } x < \frac{\umin}{\p^2}\\
v + \p x & \text{otherwise}
\end{cases}
\]
for some constant $\p > 0$.

\section{Proof}
This proof proceeds in two steps:
\begin{enumerate}
\item Use the barrier function theorem to show that $B(x,v) \leq 0$ is an invariant of the system.
\item Use the above invariant to show that $x \leq 0$ is an invariant of the system.
\end{enumerate}

\paragraph*{Step 1}
The Lie derivative of $B(x,v)$ is as follows:
\[\dot{B}(x,v) =
\begin{cases}
u + \frac{\umin v}{\sqrt{2\umin(-x + \frac{\umin}{2\p^2})}} & \text{if } x < \frac{\umin}{\p^2}\\
u + \p v & \text{otherwise}
\end{cases}
\]

Applying the exponential barrier certificate theorem results in the following constraints on $u$:
\[
u \leq
\begin{cases}
-\q(v - \sqrt{2\umin(-x + \frac{\umin}{2\p^2})}) - \frac{\umin v}{\sqrt{2\umin(-x + \frac{\umin}{2\p^2})}} & \text{if } x < \frac{\umin}{\p^2}\\
-(\q + \p) v -\q\p x & \text{otherwise}
\end{cases}
\]

We have division by a variable in the first branch of the piecewise function, but due to
the condition on that branch ($x < \frac{\umin}{\p^2}$), the denominator is
at least $\frac{|\umin|}{p} > 0$, so that term is bounded.

\paragraph*{Step 2}


\section{Exponential barrier functions}
The proof uses a general theorem about barrier functions. A barrier
function is a scalar function along trajectories of the system that is used
to establish an invariant. This particular flavor of barrier functions
establishes invariants of the form $B(\vecbold{x}) \leq 0$ where $B$ is the
barrier function. The theorem states that such a predicate is an invariant
of the system if the value of $B$ along trajectories of the system
approaches the boundary of the invariant no faster than an
exponential. More precisely:

\begin{theorem}
For an $n$-dimensional system, if there exists a function $B \in \mathcal{R}^n \rightarrow \mathcal{R}$ and some constant $\lambda \in \mathcal{R}$ such that
\[\dot{B}(\vecbold{x}) \leq \lambda B(\vecbold{x})\]
then the set $\{\vecbold{x} \in \mathcal{R}^n : B(\vecbold{x}) \leq 0\}$ is forward invariant. In other words, along all trajectories of the system $\vecbold{x}(t)$, if $B(\vecbold{x}(0)) \leq 0$, then $\forall t \geq 0, B(\vecbold{x}(t)) \leq 0$.
\label{thm:exp-barrier}
\end{theorem}

In this theorem, $\dot{B}(\vecbold{x})$ is the derivative of $B$ with
respect to time along trajectories of the system, also known as the Lie
derivative.

\end{document}
