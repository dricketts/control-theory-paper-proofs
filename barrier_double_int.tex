\documentclass[12pt]{article}

\usepackage{amsmath}

\newtheorem{theorem}{Theorem}

\title{Safety proof for double integrator}
\author{Daniel Ricketts}

\newcommand{\vecbold}[1]{\boldsymbol{#1}}
\newcommand{\umin}{u_{min}}

\begin{document}
\maketitle

\section{Problem}
We have a continuous two dimensional system with control input $u$:
\begin{align}
\dot{x} &= v \\
\dot{v} &= u
\end{align}
We want to show that $u \leq -2v-x$ enforces the invariant $x \leq 0$.

\section{Exponential barrier functions}
The proof uses a general theorem about barrier functions. A barrier
function is a scalar function along trajectories of the system that is used
to establish an invariant. This particular flavor of barrier functions
establishes invariants of the form $B(\vecbold{x}) \leq 0$ where $B$ is the
barrier function. The theorem states that such a predicate is an invariant
of the system if the value of $B$ along trajectories of the system
approaches the boundary of the invariant no faster than an
exponential. More precisely:

\begin{theorem}
For an $n$-dimensional system, if there exists a function $B \in \mathcal{R}^n \rightarrow \mathcal{R}$ and some constant $\lambda \in \mathcal{R}$ such that
\[\dot{B}(\vecbold{x}) \leq \lambda B(\vecbold{x})\]
then the set $\{\vecbold{x} \in \mathcal{R}^n : B(\vecbold{x}) \leq 0\}$ is forward invariant. In other words, along all trajectories of the system $\vecbold{x}(t)$, if $B(\vecbold{x}(0)) \leq 0$, then $\forall t \geq 0, B(\vecbold{x}(t)) \leq 0$.
\label{thm:exp-barrier}
\end{theorem}

In this theorem, $\dot{B}(\vecbold{x})$ is the derivative of $B$ with
respect to time along trajectories of the system, also known as the Lie
derivative.

\section{Proof}
The high level proof consists of two steps:
\begin{enumerate}
\item Use Theorem~\ref{thm:exp-barrier} to prove that $v \leq -x$ is an invariant of the system.
\item Use Theorem~\ref{thm:exp-barrier} and the invariance of $v \leq -x$ to prove invariance of $x \leq 0$.
\end{enumerate}

\paragraph*{Proof of 1:}
Apply Theorem~\ref{thm:exp-barrier} with $B(x,v) = v + x$. This leaves us
to prove $\dot{B}(x,v) = u + v <= \lambda (v + x)$ for some $\lambda \in
\mathcal{R}$. Setting $\lambda = -1$ results in the contraint $u <= -2v -
x$. This is exactly the constraint we impose on $u$.

\paragraph*{Proof of 2:}
Apply Theorem~\ref{thm:exp-barrier} with $B(x,v) = x$. This leaves us with
$\dot{B}(x,v) = v \leq \lambda x$. Again setting $\lambda = -1$ results in $v + x \leq 0$,
which we've already proven is an invariant of the system (step 1).

\section{Input constraint}
Now we would like to add to our system the input constraint $u \geq \umin$
for some constant $\umin < 0$. We have already imposed the input constraint
$u \leq -2v-x$ in order to enforce $x \leq 0$. Thus, in order to make sure
that there is always an available input satisfying both constraints, we
need to maintain the invariant $-2v-x \geq \umin$. We can apply
Theorem~\ref{thm:exp-barrier} using the barrier function $B(x,v) = \umin
+2v +x$. This results in the constraint $u \leq \frac{-\umin -3v -x}{2}$.

\end{document}
